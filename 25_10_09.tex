\documentclass[12pt, a4paper]{article}
\usepackage[czech]{babel}
\usepackage[utf8]{inputenc}
\usepackage{amsmath, amssymb, amsfonts}
\usepackage{graphicx}
\usepackage{geometry}
\geometry{a4paper, margin=1in}
\usepackage{parskip}
\usepackage{hyperref}
\hypersetup{colorlinks=true, linkcolor=blue}

\title{Studijní Materiál: Systémy lineárních ODR}
\author{Zpracováno z poznámek: 9.10.}
\date{\today}

\begin{document}
\maketitle
\tableofcontents
\newpage

\section{Úvod do lineárních systémů}

Uvažujme systém lineárních diferenciálních rovnic 1. řádu s konstantními koeficienty:
$$ X' = AX $$
kde $X(t)$ je vektor neznámých funkcí a $A$ je čtvercová matice $n \times n$.

\textbf{Analogie se skalární rovnicí:}
\begin{itemize}
    \item Skalární rovnice: $y' = ay \implies y(t) = C e^{at}$.
    \item Systém rovnic: $X' = AX \implies X(t) = e^{tA} C$ (kde $e^{tA}$ je maticová exponenciála).
\end{itemize}

\section{Řešení pomocí vlastních čísel a vektorů}

Hledáme řešení ve tvaru $X(t) = e^{\lambda t} v$. Dosazením do $X' = AX$ získáme podmínku:
$$ A v = \lambda v $$
To znamená, že $\lambda$ je vlastní číslo a $v$ je vlastní vektor matice $A$.

\subsection{Příklad 1: Reálná různá vlastní čísla}
\textbf{Zadání:}
$$ X' = \begin{pmatrix} 10 & -6 \\ 18 & -11 \end{pmatrix} X $$

1. **Charakteristická rovnice:**
   $$ \det(A - \lambda I) = \begin{vmatrix} 10-\lambda & -6 \\ 18 & -11-\lambda \end{vmatrix} = 0 $$
   $$ (10-\lambda)(-11-\lambda) - (-108) = \lambda^2 + \lambda - 110 + 108 = \lambda^2 + \lambda - 2 = 0 $$
   $$ (\lambda + 2)(\lambda - 1) = 0 $$
   Vlastní čísla: $\lambda_1 = 1, \quad \lambda_2 = -2$.

2. **Vlastní vektory:**
   \begin{itemize}
       \item Pro $\lambda_1 = 1$:
       $$ (A - I)v = \begin{pmatrix} 9 & -6 \\ 18 & -12 \end{pmatrix} \begin{pmatrix} v_x \\ v_y \end{pmatrix} = \begin{pmatrix} 0 \\ 0 \end{pmatrix} $$
       Rovnice $9v_x - 6v_y = 0 \implies 3v_x = 2v_y$. Volíme $v_1 = \begin{pmatrix} 2 \\ 3 \end{pmatrix}$.
       
       \item Pro $\lambda_2 = -2$:
       $$ (A + 2I)v = \begin{pmatrix} 12 & -6 \\ 18 & -9 \end{pmatrix} v = 0 $$
       Rovnice $2v_x - v_y = 0 \implies v_y = 2v_x$. Volíme $v_2 = \begin{pmatrix} 1 \\ 2 \end{pmatrix}$.
   \end{itemize}

3. **Obecné řešení:**
   $$ X(t) = C_1 e^{t} \begin{pmatrix} 2 \\ 3 \end{pmatrix} + C_2 e^{-2t} \begin{pmatrix} 1 \\ 2 \end{pmatrix} $$

\textbf{Poznámka k fundamentálnímu systému:}
Vektory $X_1(t) = e^{t} \begin{pmatrix} 2 \\ 3 \end{pmatrix}$ a $X_2(t) = e^{-2t} \begin{pmatrix} 1 \\ 2 \end{pmatrix}$ jsou lineárně nezávislá řešení.
Tyto vektory tvoří tzv. \textbf{fundamentální systém řešení}. Fundamentální matice $\Phi(t)$ by měla tyto vektory jako své sloupce.

\section{Maticová exponenciála $e^{tA}$}

\subsection{Definice a vlastnosti}
Maticová exponenciála je definována mocninnou řadou (analogicky k $e^x$):
$$ e^{tA} = \sum_{k=0}^{\infty} \frac{t^k A^k}{k!} = I + tA + \frac{t^2 A^2}{2!} + \dots $$

\textbf{Klíčové vlastnosti:}
\begin{enumerate}
    \item $\frac{d}{dt} e^{tA} = A e^{tA}$
    \item $e^{(t+s)A} = e^{tA} e^{sA}$
    \item $e^{0} = I$ (jednotková matice)
    \item Je vždy invertibilní: $(e^{tA})^{-1} = e^{-tA}$
\end{enumerate}

\subsection{Řešení počáteční úlohy a vztah k vlastním vektorům}
Pro systém $X' = AX$ s podmínkou $X(0) = X_0$ je řešení dáno vzorcem:
$$ X(t) = e^{tA} X_0 $$

\textbf{Důležitý vztah pro vlastní vektor $u_j$:}
Pokud $u_j$ je vlastní vektor příslušný vlastnímu číslu $\lambda_j$, pak platí:
$$ e^{At} u_j = e^{\lambda_j t} u_j $$
\textit{Odvození pomocí řady (z poznámek):}
$$ e^{At} u_j = e^{\lambda_j t} e^{(A - \lambda_j I)t} u_j = e^{\lambda_j t} \left( I + (A-\lambda_j I)t + \frac{(A-\lambda_j I)^2 t^2}{2!} + \dots \right) u_j $$
Protože $(A-\lambda_j I)u_j = 0$ (z definice vlastního vektoru), všechny členy řady kromě prvního vypadnou.
$$ = e^{\lambda_j t} (I \cdot u_j + 0 + \dots) = e^{\lambda_j t} u_j $$

\section{Výpočet $e^{tA}$}

\subsection{Metoda 1: Jordanův tvar (Násobná vlastní čísla)}
Pokud má matice násobná vlastní čísla a nelze ji diagonalizovat, použijeme rozklad $e^{tA} = e^{\lambda t} e^{Nt}$.

\textbf{Příklad 2:}
$$ A = \begin{pmatrix} 1 & 1 \\ 0 & 1 \end{pmatrix} $$
Vlastní číslo $\lambda = 1$ (násobnost 2). Matice $A = I + N$, kde $N = \begin{pmatrix} 0 & 1 \\ 0 & 0 \end{pmatrix}$.
$$ e^{tA} = e^{tI} e^{tN} = e^t \begin{pmatrix} 1 & t \\ 0 & 1 \end{pmatrix} = \begin{pmatrix} e^t & t e^t \\ 0 & e^t \end{pmatrix} $$

\subsection{Metoda 2: Diagonalizace}
Pokud je $A$ diagonalizovatelná, platí $A = W \Lambda W^{-1}$, a tedy:
$$ e^{tA} = W e^{t\Lambda} W^{-1} = W \begin{pmatrix} e^{\lambda_1 t} & 0 \\ 0 & e^{\lambda_2 t} \end{pmatrix} W^{-1} $$
(V poznámkách značeno také jako $W \Lambda(t) W^{-1}$, kde $\Lambda(t) = \text{diag}(e^{\lambda_i t})$).

\textbf{Příklad 3 (kompletní výpočet):}
Systém: $x' = -2x - 3y, \quad y' = 6x + 7y$. Matice $A = \begin{pmatrix} -2 & -3 \\ 6 & 7 \end{pmatrix}$.
\begin{enumerate}
    \item **Vlastní čísla:** $\lambda_1 = 1, \lambda_2 = 4$.
    \item **Vlastní vektory (tvoří matici $W$):**
    \begin{itemize}
        \item Pro $\lambda_1 = 1$: $\begin{pmatrix} -3 & -3 \\ 6 & 6 \end{pmatrix} \implies v_1 = \begin{pmatrix} 1 \\ -1 \end{pmatrix}$.
        \item Pro $\lambda_2 = 4$: $\begin{pmatrix} -6 & -3 \\ 6 & 3 \end{pmatrix} \implies v_2 = \begin{pmatrix} 1 \\ -2 \end{pmatrix}$.
    \end{itemize}
    \item **Sestavení matic:**
    $$ W = \begin{pmatrix} 1 & 1 \\ -1 & -2 \end{pmatrix}, \quad \Lambda(t) = \begin{pmatrix} e^t & 0 \\ 0 & e^{4t} \end{pmatrix} $$
    \item **Inverzní matice $W^{-1}$:**
    $$ \det(W) = 1(-2) - 1(-1) = -1 $$
    $$ W^{-1} = \frac{1}{-1} \begin{pmatrix} -2 & -1 \\ 1 & 1 \end{pmatrix} = \begin{pmatrix} 2 & 1 \\ -1 & -1 \end{pmatrix} $$
    \item **Výpočet $e^{tA}$:**
    $$ e^{tA} = \begin{pmatrix} 1 & 1 \\ -1 & -2 \end{pmatrix} \begin{pmatrix} e^t & 0 \\ 0 & e^{4t} \end{pmatrix} \begin{pmatrix} 2 & 1 \\ -1 & -1 \end{pmatrix} $$
    $$ = \begin{pmatrix} e^t & e^{4t} \\ -e^t & -2e^{4t} \end{pmatrix} \begin{pmatrix} 2 & 1 \\ -1 & -1 \end{pmatrix} $$
    $$ = \begin{pmatrix} 2e^t - e^{4t} & e^t - e^{4t} \\ -2e^t + 2e^{4t} & -e^t + 2e^{4t} \end{pmatrix} $$
\end{enumerate}

\section{Fundamentální matice $\Phi(t)$}

Fundamentální matice $\Phi(t)$ je matice, jejíž sloupce tvoří lineárně nezávislá řešení systému.
Obecné řešení lze zapsat jako $X(t) = \Phi(t) K$.

\textbf{Vztah k maticové exponenciále:}
$$ e^{tA} = \Phi(t) \Phi(0)^{-1} $$
Pokud zvolíme počáteční podmínku v $t=0$ tak, že $\Phi(0) = I$, pak přímo $\Phi(t) = e^{tA}$.

\textbf{Výpočet konstanty $K$ z počátečních podmínek:}
Pokud máme zadán počáteční vektor $X(0) = I$ (v poznámkách značeno $I$ nebo $L$), pak konstantu $K$ určíme vztahem:
$$ K = \Phi^{-1}(0) \cdot I $$

\section{Cayley-Hamiltonova věta a polynomy}
Každá čtvercová matice je kořenem svého charakteristického polynomu $P(\lambda) = \det(A - \lambda I) = 0$.
$$ P(A) = 0 $$
Například pro $A = \begin{pmatrix} 1 & 1 \\ 0 & 1 \end{pmatrix}$ je $P(\lambda) = (\lambda-1)^2 = \lambda^2 - 2\lambda + 1$.
Platí: $A^2 - 2A + I = 0$, tedy $A^2 = 2A - I$.
Tato vlastnost umožňuje vyjádřit $e^{tA}$ jako polynom matice $A$ stupně $n-1$:
$$ e^{tA} = b_0(t) I + b_1(t) A $$

\end{document}