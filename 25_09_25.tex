\documentclass[12pt, a4paper]{article}
\usepackage[czech]{babel}
\usepackage[utf8]{inputenc}
\usepackage{amsmath, amssymb, amsfonts}
\usepackage{graphicx}
\usepackage{geometry}
\geometry{a4paper, margin=1in}
\usepackage{parskip} % Lepší formátování odstavců
\usepackage{hyperref}
\hypersetup{colorlinks=true, linkcolor=blue}

% --- Definice pro obrázky ---
% Odkomentujte a upravte název souboru, pokud chcete vložit obrázek
% \newcommand{\GrafSeparace}{\includegraphics[width=0.45\textwidth]{separace_graf.png}}

\title{Studijní Materiál: Metody řešení ODR 1. řádu}
\author{Zpracováno z poznámek: 25.9.}
\date{\today}

\begin{document}
\maketitle
\tableofcontents
\newpage

\section{Separace proměnných}

Tato metoda se používá pro rovnice tvaru $y' = g(x)h(y)$.

\textbf{Obecný postup:}
\begin{enumerate}
    \item Najdeme konstantní řešení (kde $h(y)=0$).
    \item Separujeme proměnné: $\frac{1}{h(y)} dy = g(x) dx$.
    \item Integrujeme obě strany: $\int \frac{1}{h(y)} dy = \int g(x) dx + C$.
    \item Pokud je to možné, vyjádříme $y$ explicitně.
\end{enumerate}

\subsection{Příklad 1: $y' = x(y-1)$}
\textbf{Zadání:} Nalezněte řešení splňující počáteční podmínku $y(-20) = 0$.

1. **Separace:**
   $$ \frac{dy}{y-1} = x dx $$
   (Všimneme si, že $y=1$ je konstantní řešení, ale to nesplňuje podmínku $y(-20)=0$).

2. **Integrace:**
   $$ \int \frac{dy}{y-1} = \int x dx $$
   $$ \ln|y-1| = \frac{x^2}{2} + C $$

3. **Úprava na explicitní tvar:**
   $$ |y-1| = e^{\frac{x^2}{2} + C} = e^C \cdot e^{\frac{x^2}{2}} $$
   Položme $K = \pm e^C$.
   $$ y - 1 = K e^{\frac{x^2}{2}} \implies y = 1 + K e^{\frac{x^2}{2}} $$

4. **Dosazení počáteční podmínky $y(-20)=0$:**
   $$ 0 = 1 + K e^{\frac{(-20)^2}{2}} = 1 + K e^{200} $$
   $$ K = -e^{-200} $$

5. **Výsledek:**
   $$ y(x) = 1 - e^{-200} \cdot e^{\frac{x^2}{2}} = 1 - e^{\frac{x^2}{2} - 200} $$

\subsection{Příklad 2: $y' = xy^3$}
\textbf{Zadání:} Nalezněte řešení pro $y(0)=1$.

1. **Separace:**
   $$ \frac{dy}{y^3} = x dx \implies y^{-3} dy = x dx $$

2. **Integrace:**
   $$ \frac{y^{-2}}{-2} = \frac{x^2}{2} + C $$
   $$ -\frac{1}{2y^2} = \frac{x^2}{2} + C $$
   Vynásobíme $-2$:
   $$ \frac{1}{y^2} = -x^2 - 2C \quad (\text{označme } C_2 = -2C) $$
   $$ y^2 = \frac{1}{C_2 - x^2} $$

3. **Explicitní tvar a podmínka:**
   $$ y = \pm \sqrt{\frac{1}{C_2 - x^2}} $$
   Pro $y(0)=1$ volíme kladnou větev:
   $$ 1 = \sqrt{\frac{1}{C_2 - 0}} \implies C_2 = 1 $$
   **Výsledek:** $y(x) = \sqrt{\frac{1}{1-x^2}}$, definováno pro $x \in (-1, 1)$.

\section{Bernoulliho rovnice a metoda $y=uv$}

Pro rovnice, které nejsou separovatelné, můžeme zkusit substituci $y(x) = u(x)v(x)$.

\subsection{Příklad 3: $y' - \frac{y}{x} = 2x^3$}
Jde o lineární rovnici, kterou v poznámkách řešíme metodou součinu funkcí (Bernoulliho metodou).

1. **Dosazení $y=uv$:**
   $$ y' = u'v + uv' $$
   $$ (u'v + uv') - \frac{uv}{x} = 2x^3 $$
   $$ u'v + u(v' - \frac{v}{x}) = 2x^3 $$

2. **Nulování závorky (hledáme $v$):**
   $$ v' - \frac{v}{x} = 0 \implies \frac{v'}{v} = \frac{1}{x} $$
   $$ \ln|v| = \ln|x| \implies v = x $$

3. **Výpočet $u$ (dosadíme $v=x$):**
   $$ u' \cdot x = 2x^3 $$
   $$ u' = 2x^2 $$
   $$ u = \int 2x^2 dx = \frac{2}{3}x^3 + C $$

4. **Celkové řešení:**
   $$ y = u \cdot v = \left(\frac{2}{3}x^3 + C\right) x = \frac{2}{3}x^4 + Cx $$

\section{Lineární ODR 1. řádu}

Tvar: $y' + a(x)y = g(x)$.

\subsection{Metoda 1: Variace konstant}
Postup:
1. Vyřešíme homogenní rovnici $y' + a(x)y = 0$ (separací).
2. Konstantu $C$ nahradíme funkcí $C(x)$ a dosadíme do původní rovnice.

\subsubsection{Příklad 4: $xy' + 3y = x^2$}
Upravíme dělením $x$: $y' + \frac{3}{x}y = x$.

1. **Homogenní řešení:**
   $$ y' = -\frac{3}{x}y \implies \ln|y| = -3\ln|x| \implies y_h = \frac{C}{x^3} $$

2. **Variace:**
   Hledáme $y = \frac{C(x)}{x^3}$.
   Derivace: $y' = C'(x)x^{-3} - 3C(x)x^{-4}$.
   Dosazení do $xy' + 3y = x^2$:
   $$ x(C'x^{-3} - 3Cx^{-4}) + 3(Cx^{-3}) = x^2 $$
   $$ C'x^{-2} - 3Cx^{-3} + 3Cx^{-3} = x^2 $$
   $$ C'x^{-2} = x^2 \implies C' = x^4 $$
   $$ C(x) = \frac{x^5}{5} + K $$

3. **Výsledek:**
   $$ y(x) = \left(\frac{x^5}{5} + K\right)\frac{1}{x^3} = \frac{x^2}{5} + \frac{K}{x^3} $$

\subsection{Metoda 2: Integrační faktor}
Celou rovnici vynásobíme faktorem $\mu(x) = e^{\int a(x) dx}$, čímž převedeme levou stranu na derivaci součinu.

\subsubsection{Příklad 5: $y' + \frac{1}{x}y = \frac{1}{x^2}$}
1. **Faktor:** $a(x) = \frac{1}{x} \implies \mu(x) = e^{\ln|x|} = x$.
2. **Násobení:**
   $$ x \cdot y' + 1 \cdot y = \frac{1}{x} $$
   $$ (xy)' = \frac{1}{x} $$
3. **Integrace:**
   $$ xy = \ln|x| + C \implies y = \frac{\ln|x| + C}{x} $$

\subsubsection{Příklad 6: Goniometrická rovnice}
Zadání: $y' + y \tan x = 2 \sin x \cos x$ (s podmínkou $y(0)=...$).

1. **Faktor:**
   $$ a(x) = \tan x = \frac{\sin x}{\cos x} $$
   $$ \int \frac{\sin x}{\cos x} dx = -\ln|\cos x| $$
   $$ \mu(x) = e^{-\ln|\cos x|} = \frac{1}{\cos x} $$

2. **Násobení rovnice faktorem $\frac{1}{\cos x}$:**
   $$ y' \frac{1}{\cos x} + y \frac{\sin x}{\cos^2 x} = \frac{2 \sin x \cos x}{\cos x} $$
   Levá strana je derivace součinu $(y \cdot \frac{1}{\cos x})'$.
   $$ \left( \frac{y}{\cos x} \right)' = 2 \sin x $$

3. **Integrace:**
   $$ \frac{y}{\cos x} = \int 2 \sin x dx = -2 \cos x + C $$
   $$ y(x) = -2 \cos^2 x + C \cos x $$

\end{document}