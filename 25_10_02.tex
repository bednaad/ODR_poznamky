\documentclass[12pt, a4paper]{article}
\usepackage[czech]{babel}
\usepackage[utf8]{inputenc}
\usepackage{amsmath, amssymb, amsfonts}
\usepackage{graphicx}
\usepackage{geometry}
\geometry{a4paper, margin=1in}
\usepackage{parskip} % Pro hezčí formátování bez odsazování prvních řádků
\usepackage{hyperref}
\hypersetup{colorlinks=true, linkcolor=blue}

% --- Definice pro obrázky ---
% Pro kompilaci odkomentujte řádky uvnitř figure a ujistěte se, že máte soubory ve složce.
\newcommand{\GrafYAbs}{\includegraphics[width=0.45\textwidth]{y_abs.png}}
\newcommand{\GrafStabilita}{\includegraphics[width=0.6\textwidth]{stabilita_diagram.png}}

\title{Studijní Materiál: Obyčejné Diferenciální Rovnice}
\author{Zpracováno z poznámek (2.10., 25.9., 27.11.)}
\date{\today}

\begin{document}
\maketitle
\tableofcontents
\newpage

\section{Teorie: Existence a Jednoznačnost}

\subsection{Picard-Lindelöfova věta}
Uvažujme počáteční problém:
$$ y' = f(x, y), \quad y(x_0) = y_0 $$
Nechť $G \subseteq \mathbb{R}^2$ je oblast obsahující bod $(x_0, y_0)$.
\begin{itemize}
    \item \textbf{Existence:} Pokud je $f(x, y)$ spojitá na $G$, pak existuje alespoň jedno řešení definované na nějakém intervalu $I$ obsahujícím $x_0$.
    \item \textbf{Jednoznačnost:} Pokud je navíc parciální derivace $\frac{\partial f}{\partial y}$ spojitá na $G$, pak existuje právě jedno \textbf{maximální} řešení.
\end{itemize}
\textit{Poznámka: Maximální řešení je takové, které nelze prodloužit na větší interval.}

\subsection{Protipříklad (Porušení jednoznačnosti)}
$$ y' = \sqrt{|y|} $$
\begin{itemize}
    \item Funkce $f(x,y) = \sqrt{|y|}$ je spojitá všude.
    \item Derivace $\frac{\partial f}{\partial y} = \frac{\text{sgn}(y)}{2\sqrt{|y|}}$ není definována pro $y=0$.
    \item **Důsledek:** Bodem $(x_0, 0)$ může procházet více řešení (např. triviální $y=0$ a parabolické větve).
\end{itemize}

% \begin{figure}[h]
%     \centering
%     %\GrafYAbs
%     \caption{Ilustrace nejednoznačnosti řešení v bodě y=0.}
% \end{figure}

\section{Metody řešení ODR 1. řádu}

\subsection{Separace proměnných}
Tvar: $y' = g(x)h(y)$.
Postup: $\int \frac{dy}{h(y)} = \int g(x) dx$.

\subsubsection{Příklad 1: $y' = |y|$}
Pro $y > 0$:
$$ \int \frac{dy}{y} = \int dx \implies \ln|y| = x + C \implies y = K e^x $$
Pro $y < 0$:
$$ \int \frac{dy}{-y} = \int dx \implies -\ln|y| = x + C \implies y = K e^{-x} $$

\subsubsection{Příklad 2: $y' = \sqrt{1-y^2}$}
$$ \int \frac{dy}{\sqrt{1-y^2}} = \int dx $$
$$ \arcsin y = x + C $$
$$ y = \sin(x+C) $$
Podmínka existence: $x+C \in (-\frac{\pi}{2}, \frac{\pi}{2})$.

\subsection{Lineární rovnice a metoda $y=uv$}

\subsubsection{Příklad: $xy' - y = 3x^2, \quad y(0)=1$}
Řešíme pomocí substituce $y = u(x) \cdot v(x)$.
Derivace součinu je $y' = u'v + uv'$.

1. **Dosazení do rovnice:**
   $$ x(u'v + uv') - uv = 3x^2 $$
   Roznásobíme a vytkneme $u$ ze členů obsahujících $v$:
   $$ x u' v + x u v' - uv = 3x^2 $$
   $$ x u' v + u(\underbrace{xv' - v}_{=0}) = 3x^2 $$

2. **Výpočet $v(x)$ (řešení závorky):**
   Položíme závorku rovnu nule:
   $$ xv' - v = 0 \implies x \frac{dv}{dx} = v $$
   Separace proměnných:
   $$ \frac{dv}{v} = \frac{dx}{x} \implies \ln|v| = \ln|x| \implies v = x $$

3. **Výpočet $u(x)$ (dosazení $v$):**
   Vrátíme se k rovnici $x u' v = 3x^2$ a dosadíme $v=x$:
   $$ x \cdot u' \cdot x = 3x^2 $$
   $$ x^2 u' = 3x^2 \quad (\text{zde se často chybuje v mocninách}) $$
   Vydělíme $x^2$ (pro $x \neq 0$):
   $$ u' = 3 $$
   Integrací získáme $u$:
   $$ u = \int 3 \, dx = 3x + C $$

4. **Obecné řešení ($y=uv$):**
   $$ y(x) = (3x + C) \cdot x = 3x^2 + Cx $$

5. **Počáteční podmínka $y(0)=1$:**
   Dosadíme $x=0$:
   $$ y(0) = 3(0)^2 + C(0) = 0 $$
   To se nerovná zadané hodnotě $1$.
   \textbf{Závěr:} Úloha nemá řešení (singulární bod v $x=0$, integrální křivky procházejí počátkem 0, nikoliv bodem [0,1]).

\section{Aplikace ODR (Fyzika a Populace)}

\subsection{Dynamika (Odpor vzduchu)}
Z Newtonova zákona $F = ma$:
\begin{enumerate}
    \item **Pouze odpor** ($F = -kv$):
    $$ m v' = -kv \implies v(t) = v_0 e^{-\frac{k}{m}t} $$
    \item **Odpor + Gravitace** ($F = -kv - mg$):
    $$ v' + \frac{k}{m}v = -g $$
    Řešení: $v(t) = -\frac{mg}{k} + C e^{-\frac{k}{m}t}$.
\end{enumerate}

\subsection{Míchací model (Nádrž)}
$$ V \cdot c' = q_{in} C_{in} - q_{out} c(t) $$
Pokud $q_{in} = q_{out} = q$:
$$ c' + \frac{q}{V}c = \frac{q}{V}C_{in} $$

\subsection{Malthusův model (Exponenciální růst)}
Základní rovnice: Rychlost růstu je úměrná velikosti populace.
$$ P'(t) = k P(t) $$
Řešení:
$$ P(t) = P_0 e^{kt} $$

\subsubsection{Konkrétní příklad z poznámek (výpočet k)}
Zadání: Populace vzrostla o 10 \% za 10 hodin. Jaké je $k$?
$$ P(10) = 1.1 P_0 $$
$$ P_0 e^{k \cdot 10} = 1.1 P_0 $$
$$ e^{10k} = 1.1 $$
$$ 10k = \ln(1.1) \implies k = \frac{\ln(1.1)}{10} \approx 0.009531 $$

\subsubsection{Aplikace na COVID (data z poznámek)}
Poznámky uvádí data z roku 2020:
\begin{itemize}
    \item 22.8.: 234 případů ($P_0$)
    \item Předpoklad: Exponenciální růst s konstantou $k \approx 0.00953$ (nebo podobnou vypočtenou z dat).
    \item Predikce pro 21.9. (cca 30 dní):
    $$ P(30) = 234 \cdot e^{0.00953 \cdot 30} \approx \dots $$
    (Poznámky obsahují konkrétní numerické výpočty predikcí pro různé dny).
\end{itemize}

\subsection{Logistický model (Omezený růst)}
Růst je brzděn kapacitou prostředí ($K=1$).
$$ y' = y(1-y) $$
\textbf{Řešení:}
1. Separace: $\int \frac{dy}{y(1-y)} = \int dx$.
2. Rozklad na parciální zlomky: $\frac{1}{y(1-y)} = \frac{1}{y} + \frac{1}{1-y}$.
3. Integrace:
   $$ \ln|y| - \ln|1-y| = x + C $$
   $$ \ln \left| \frac{y}{1-y} \right| = x + C $$
4. Odlogaritmování a úprava:
   $$ y(x) = \frac{K e^{x}}{1 + K e^{x}} $$
   (Grafem je tzv. S-křivka / sigmoida).

\section{Kvalitativní teorie - Stabilita}

Uvažujme autonomní rovnici $y' = f(y)$.
Stacionární body jsou řešení rovnice $f(y) = 0$.

\subsection{Určení stability z fázové přímky}
Analyzujeme znaménko $f(y)$ v okolí stacionárního bodu $c$:
\begin{enumerate}
    \item \textbf{Stabilní (II):}
    \begin{itemize}
        \item Nalevo od $c$: $f(y) > 0$ (řešení roste k $c$).
        \item Napravo od $c$: $f(y) < 0$ (řešení klesá k $c$).
        \item Šipky směřují **k bodu**.
    \end{itemize}
    
    \item \textbf{Nestabilní (I):}
    \begin{itemize}
        \item Nalevo od $c$: $f(y) < 0$ (řešení klesá pryč).
        \item Napravo od $c$: $f(y) > 0$ (řešení roste pryč).
        \item Šipky směřují **od bodu**.
    \end{itemize}
    
    \item \textbf{Semistabilní:} Znaménko se nemění (např. inflexní bod).
\end{enumerate}

\subsection{Analýza logistické rovnice $y' = y(1-y)$}
Funkce $f(y) = y - y^2$ (parabola otočená dolů).
Kořeny: $y=0$ a $y=1$.

\begin{itemize}
    \item **Bod 0:** $f(y)$ přechází z mínusu do plusu $\implies$ **Nestabilní**.
    \item **Bod 1:** $f(y)$ přechází z plusu do mínusu $\implies$ **Stabilní**.
\end{itemize}

% \begin{figure}[h]
%     \centering
%     %\GrafStabilita
%     \caption{Fázový portrét logistické rovnice: 0 je nestabilní, 1 je stabilní.}
% \end{figure}

\end{document}