\documentclass[12pt, a4paper]{article}
\usepackage[czech]{babel}
\usepackage[utf8]{inputenc}
\usepackage{amsmath, amssymb, amsfonts}
\usepackage{graphicx}
\usepackage{geometry}
\geometry{a4paper, margin=1in}
\usepackage{parskip}
\usepackage{hyperref}
\hypersetup{colorlinks=true, linkcolor=blue}

% --- Místo pro obrázky ---
% \newcommand{\ObrSektory}{\includegraphics[width=0.8\textwidth]{sektory_tabule.jpg}}
% \newcommand{\ObrNewton}{\includegraphics[width=0.8\textwidth]{newton_potencial.jpg}}

\title{Studijní Materiál: Topologie fázového portrétu a Konzervativní systémy}
\author{Zpracováno z poznámek: 20.11.}
\date{\today}

\begin{document}
\maketitle
\tableofcontents
\newpage

\section{Topologie okolí bodu rovnováhy}

Zkoumáme chování trajektorií $\Gamma(t)$ v okolí bodu rovnováhy $x_0$, kde $F(x_0)=0$.
Zatímco u hyperbolických bodů (sedlo, uzel, ohnisko) je situace jasná z linearizace, u nehyperbolických bodů (např. vlastní čísla s nulovou reálnou částí) je struktura složitější.

\subsection{Sektory a Separatrixy}
Okolí bodu rovnováhy může být rozděleno tzv. **separatrixami** (speciální trajektorie $\Gamma$, které pro $t \to \pm \infty$ konvergují přímo k bodu rovnováhy) na sektory s odlišným chováním.

Rozlišujeme tři základní typy sektorů:
\begin{enumerate}
    \item **Hyperbolický sektor:**
    Trajektorie $\Gamma$ v tomto sektoru se k bodu rovnováhy přiblíží, "minou ho" a opět se vzdálí. (Typické pro okolí sedla).
    
    \item **Parabolický sektor:**
    Všechny trajektorie $\Gamma$ v tomto sektoru konvergují k bodu rovnováhy (pro $t \to \infty$) nebo z něj všechny divergují (pro $t \to -\infty$). Chová se to jako výseč uzlu.
    
    \item **Eliptický sektor:**
    Trajektorie $\Gamma$ vycházejí z bodu rovnováhy, opíší smyčku a vracejí se zpět do bodu rovnováhy (homoklinické orbity). Celý sektor je vyplněn uzavřenými smyčkami.
\end{enumerate}

% \begin{figure}[h]
%     \centering
%     %\ObrSektory
%     \caption{Klasifikace sektorů v okolí degenerovaného bodu rovnováhy.}
% \end{figure}

\subsection{Složené body rovnováhy}
Složený bod rovnováhy vzniká splynutím více jednoduchých bodů rovnováhy (např. při změně parametru).
Jeho index (topologická charakteristika) je roven součtu indexů bodů, které splynuly.
Například splynutím sedla (index -1) a uzlu (index +1) vznikne bod s indexem 0 (např. sedlo-uzel).
Okolí takového bodu se skládá z kombinace hyperbolických, parabolických a eliptických sektorů.
Vztah pro počet sektorů:
$$ 2 + E - H = 2I $$
kde $E$ je počet eliptických sektorů, $H$ počet hyperbolických sektorů a $I$ je index bodu.

\section{Analýza v polárních souřadnicích (Blow-up metoda)}

Pokud linearizace selhává (např. Jacobiho matice je nulová), je nutné zkoumat nelineární členy. Často pomůže transformace do polárních souřadnic, která "nafoukne" bod rovnováhy na kružnici.

\subsection{Odvození transformačních vztahů}
Mějme systém $\dot{x} = P(x,y), \dot{y} = Q(x,y)$.
Zavedeme substituci:
$$ x = r \cos \varphi, \quad y = r \sin \varphi $$
Z toho plyne $r^2 = x^2 + y^2$ a $\tan \varphi = \frac{y}{x}$.

**1. Rovnice pro $r$:**
Derivujeme $r^2$:
$$ 2r\dot{r} = 2x\dot{x} + 2y\dot{y} \implies \dot{r} = \frac{x}{r}\dot{x} + \frac{y}{r}\dot{y} $$
Dosadíme za $x, y$:
$$ \dot{r} = \dot{x} \cos \varphi + \dot{y} \sin \varphi $$

**2. Rovnice pro $\varphi$:**
Derivujeme $\tan \varphi = y/x$:
$$ \frac{1}{\cos^2 \varphi} \dot{\varphi} = \frac{\dot{y}x - y\dot{x}}{x^2} $$
Vynásobíme $\cos^2 \varphi = (x/r)^2$:
$$ \dot{\varphi} = \frac{x\dot{y} - y\dot{x}}{r^2} = \frac{1}{r} (\dot{y} \cos \varphi - \dot{x} \sin \varphi) $$

\subsection{Příklad z tabule}
Uvažujme systém, který po převodu dává rovnice typu:
$$ \dot{r} = r^k f(\varphi) $$
$$ \dot{\varphi} = r^{k-1} g(\varphi) $$

Konkrétní analýza na tabuli (zřejmě pro určitý systém) ukazuje hledání směrů, kde $\dot{\varphi} = 0$ (invariantní paprsky).
* Pokud pro nějaké $\varphi_0$ platí $\dot{\varphi}=0$, pak polopřímka s úhlem $\varphi_0$ je invariantní (trajektorie $\Gamma$ se pohybuje po přímce).
* Na této přímce rozhoduje znaménko $\dot{r}$:
    * $\dot{r} < 0$: Trajektorie $\Gamma$ jde do počátku (stabilní směr).
    * $\dot{r} > 0$: Trajektorie $\Gamma$ jde od počátku (nestabilní směr).
* Tímto způsobem určíme, zda se v daném sektoru mezi paprsky trajektorie chovají jako u sedla, uzlu atd.

\section{Hamiltonovské systémy}

Konzervativní systémy, kde nedochází ke ztrátě energie (neexistuje tření).

\subsection{Definice a Hamiltonián}
Systém je Hamiltonovský, pokud existuje funkce $H(x,y)$ taková, že:
$$ \dot{x} = \frac{\partial H}{\partial y}, \quad \dot{y} = -\frac{\partial H}{\partial x} $$

**Důsledek:**
$$ \frac{dH}{dt} = \frac{\partial H}{\partial x}\dot{x} + \frac{\partial H}{\partial y}\dot{y} = \frac{\partial H}{\partial x}\left(\frac{\partial H}{\partial y}\right) + \frac{\partial H}{\partial y}\left(-\frac{\partial H}{\partial x}\right) = 0 $$
Hamiltonián $H(x,y)$ je **prvním integrálem** systému. Trajektorie $\Gamma$ leží na hladinách konstantní energie $H(x,y) = C$.

\subsection{Příklad z poznámek (dokončení)}
Systém:
$$ \dot{x} = y + x^2 - y^2 $$
$$ \dot{y} = -x - 2xy $$

**1. Nalezení Hamiltoniánu:**
Integrujeme $\dot{x}$ podle $y$:
$$ H = \int (y + x^2 - y^2) dy = \frac{y^2}{2} + x^2y - \frac{y^3}{3} + C(x) $$
Derivujeme podle $x$ a porovnáme s $-\dot{y}$:
$$ \frac{\partial H}{\partial x} = 2xy + C'(x) \quad \text{musí se rovnat} \quad -(-x - 2xy) = x + 2xy $$
$$ C'(x) = x \implies C(x) = \frac{x^2}{2} $$
Výsledný Hamiltonián:
$$ H(x,y) = \frac{1}{2}(x^2 + y^2) + x^2y - \frac{1}{3}y^3 $$

**2. Analýza bodů rovnováhy:**
Body rovnováhy systému odpovídají stacionárním bodům funkce $H(x,y)$ (kde $\nabla H = 0$).
Zde BR $[0,0]$.
Hessián funkce $H$ v $[0,0]$:
$$ \text{Hess } H = \begin{pmatrix} H_{xx} & H_{xy} \\ H_{yx} & H_{yy} \end{pmatrix}_{(0,0)} = \begin{pmatrix} 1 & 0 \\ 0 & 1 \end{pmatrix} $$
Determinant $> 0$, stopa $> 0$ (nebo vlastní čísla $1, 1 > 0$).
Bod $[0,0]$ je **lokální minimum** funkce $H$.
\textbf{Závěr:} Protože $H$ má v počátku ostré lokální minimum, jsou trajektorie $\Gamma$ v okolí uzavřené křivky obíhající počátek. Bod $[0,0]$ je **střed** (Center).

\section{Newtonovské systémy}

Fyzikální systémy popisující pohyb částice v potenciálovém poli $V(x)$.
Rovnice: $m \ddot{x} = F(x) = -V'(x)$.
Převedeno na systém 1. řádu (pro $m=1$):
$$ \dot{x} = y $$
$$ \dot{y} = -V'(x) $$

\subsection{Hamiltonián (Celková energie)}
$$ H(x,y) = \text{E}_{kin} + \text{E}_{pot} = \frac{1}{2}y^2 + V(x) $$
Trajektorie $\Gamma$ jsou dány rovnicí $\frac{1}{2}y^2 + V(x) = E$ (konstanta).
Odtud lze vyjádřit rychlost $y = \pm \sqrt{2(E - V(x))}$.
Pohyb je možný jen tam, kde $E \ge V(x)$.

\subsection{Vztah potenciálu $V(x)$ a fázového portrétu}
Z tvaru funkce $V(x)$ můžeme přímo nakreslit fázový portrét:

\begin{itemize}
    \item **Lokální minimum $V(x)$:**
    Odpovídá bodu rovnováhy typu **STŘED**.
    Částice kmitá v "dolíku" potenciálu. Fázové trajektorie $\Gamma$ jsou uzavřené smyčky kolem tohoto bodu.
    
    \item **Lokální maximum $V(x)$:**
    Odpovídá bodu rovnováhy typu **SEDLO**.
    Částice je na vrcholu kopce, což je nestabilní poloha. Separatrixy oddělují pohyb na jednu a druhou stranu kopce.
\end{itemize}

% \begin{figure}[h]
%     \centering
%     %\ObrNewton
%     \caption{Nahoře: Graf potenciálu V(x). Dole: Odpovídající fázový portrét (minima -> středy, maxima -> sedla).}
% \end{figure}

\section{Gradientní systémy}

Systémy, kde pohyb sleduje největší spád potenciálu (opak konzervativních systémů - zde se energie maximálně "ztrácí").
$$ \dot{x} = -\nabla V(x) $$

**Vlastnosti:**
\begin{itemize}
    \item Trajektorie $\Gamma$ jsou kolmé na vrstevnice $V(x)$.
    \item Funkce $V(x)$ podél řešení klesá: $\frac{dV}{dt} \le 0$.
    \item Neexistují žádné uzavřené cykly (limitní cykly).
    \item Všechna řešení konvergují k bodům rovnováhy (lokálním extrémům $V$).
    \item **Lokální minima $V$** jsou **asymptoticky stabilní** uzly.
    \item **Lokální maxima $V$** jsou **nestabilní** uzly (zdroje).
    \item **Sedlové body $V$** jsou **sedla** ve fázovém portrétu.
\end{itemize}

\end{document}